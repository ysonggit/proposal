\begin{animateinline}[
  begin={%
    \begin{tikzpicture}%
      [scale=0.75,
      post/.style={->,>=stealth', semithick, draw=blue!50},
      node/.style={circle,fill=red!20,draw,font=\sffamily\small}]%
      \useasboundingbox (0,0) rectangle (5,5.5);
    },
    end={\end{tikzpicture}}
  ]{10}
  \coordinate (A1) at (0,4);
  \coordinate (A2) at (1,4);
  \coordinate (A3) at (2.5,5);
  \coordinate (A4) at (3.5,3.5);
  \coordinate (A5) at (3,2);
  \coordinate (A6) at (1.75,2);
  \path [draw] (A1) -- (A2);
  \path [draw] (A2) -- (A3);
  \path [draw] (A3) -- (A4);
  \path [draw] (A4) -- (A5);
  \path [draw] (A5) -- (A6);
  \path [draw] (A6) -- (A1);
  \fill[blue!20] (A1) -- (A2) -- (A3) -- (A4)  -- (A5) --(A6) -- cycle;               \node[] at (1.5,3.5) {$X_{k+1}(\eta_k, u_k)$};
  \coordinate (L) at (3,3.5);
  \draw[fill=blue] (L) circle (0.1);
  \node[] at (3.7,4.2) {$H(y_k)$};
  \draw[blue] (L) circle (1.5);
  %%%% new frame %%%%
  \newframe*
  \multiframe{10}{}{ 
    \coordinate (A1) at (0,4);
    \coordinate (A2) at (1,4);
    \coordinate (A3) at (2.5,5);
    \coordinate (A4) at (3.5,3.5);
    \coordinate (A5) at (3,2);
    \coordinate (A6) at (1.75,2);
    \coordinate (L) at (3,3.5);
    
    \draw[fill=blue] (L) circle (0.1);
    \node[] at (2.5,3) {$\eta_{k+1}$};
    \begin{scope}
      \clip (A1) -- (A2) -- (A3) -- (A4)  -- (A5) -- (A6) -- cycle;
      \draw[blue, fill=blue!20] (L) circle (1.5);
    \end{scope}
    \node[] at (2.5,3.4) {$\eta_{k+1}$};
    \draw[blue] (2.55,4.95) -- (A4);
    \draw[blue] (A4) -- (A5);
  }
\end{animateinline}